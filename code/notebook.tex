
% Default to the notebook output style

    


% Inherit from the specified cell style.




    
\documentclass[11pt]{article}

    
    
    \usepackage[T1]{fontenc}
    % Nicer default font (+ math font) than Computer Modern for most use cases
    \usepackage{mathpazo}

    % Basic figure setup, for now with no caption control since it's done
    % automatically by Pandoc (which extracts ![](path) syntax from Markdown).
    \usepackage{graphicx}
    % We will generate all images so they have a width \maxwidth. This means
    % that they will get their normal width if they fit onto the page, but
    % are scaled down if they would overflow the margins.
    \makeatletter
    \def\maxwidth{\ifdim\Gin@nat@width>\linewidth\linewidth
    \else\Gin@nat@width\fi}
    \makeatother
    \let\Oldincludegraphics\includegraphics
    % Set max figure width to be 80% of text width, for now hardcoded.
    \renewcommand{\includegraphics}[1]{\Oldincludegraphics[width=.8\maxwidth]{#1}}
    % Ensure that by default, figures have no caption (until we provide a
    % proper Figure object with a Caption API and a way to capture that
    % in the conversion process - todo).
    \usepackage{caption}
    \DeclareCaptionLabelFormat{nolabel}{}
    \captionsetup{labelformat=nolabel}

    \usepackage{adjustbox} % Used to constrain images to a maximum size 
    \usepackage{xcolor} % Allow colors to be defined
    \usepackage{enumerate} % Needed for markdown enumerations to work
    \usepackage{geometry} % Used to adjust the document margins
    \usepackage{amsmath} % Equations
    \usepackage{amssymb} % Equations
    \usepackage{textcomp} % defines textquotesingle
    % Hack from http://tex.stackexchange.com/a/47451/13684:
    \AtBeginDocument{%
        \def\PYZsq{\textquotesingle}% Upright quotes in Pygmentized code
    }
    \usepackage{upquote} % Upright quotes for verbatim code
    \usepackage{eurosym} % defines \euro
    \usepackage[mathletters]{ucs} % Extended unicode (utf-8) support
    \usepackage[utf8x]{inputenc} % Allow utf-8 characters in the tex document
    \usepackage{fancyvrb} % verbatim replacement that allows latex
    \usepackage{grffile} % extends the file name processing of package graphics 
                         % to support a larger range 
    % The hyperref package gives us a pdf with properly built
    % internal navigation ('pdf bookmarks' for the table of contents,
    % internal cross-reference links, web links for URLs, etc.)
    \usepackage{hyperref}
    \usepackage{longtable} % longtable support required by pandoc >1.10
    \usepackage{booktabs}  % table support for pandoc > 1.12.2
    \usepackage[inline]{enumitem} % IRkernel/repr support (it uses the enumerate* environment)
    \usepackage[normalem]{ulem} % ulem is needed to support strikethroughs (\sout)
                                % normalem makes italics be italics, not underlines
    

    
    
    % Colors for the hyperref package
    \definecolor{urlcolor}{rgb}{0,.145,.698}
    \definecolor{linkcolor}{rgb}{.71,0.21,0.01}
    \definecolor{citecolor}{rgb}{.12,.54,.11}

    % ANSI colors
    \definecolor{ansi-black}{HTML}{3E424D}
    \definecolor{ansi-black-intense}{HTML}{282C36}
    \definecolor{ansi-red}{HTML}{E75C58}
    \definecolor{ansi-red-intense}{HTML}{B22B31}
    \definecolor{ansi-green}{HTML}{00A250}
    \definecolor{ansi-green-intense}{HTML}{007427}
    \definecolor{ansi-yellow}{HTML}{DDB62B}
    \definecolor{ansi-yellow-intense}{HTML}{B27D12}
    \definecolor{ansi-blue}{HTML}{208FFB}
    \definecolor{ansi-blue-intense}{HTML}{0065CA}
    \definecolor{ansi-magenta}{HTML}{D160C4}
    \definecolor{ansi-magenta-intense}{HTML}{A03196}
    \definecolor{ansi-cyan}{HTML}{60C6C8}
    \definecolor{ansi-cyan-intense}{HTML}{258F8F}
    \definecolor{ansi-white}{HTML}{C5C1B4}
    \definecolor{ansi-white-intense}{HTML}{A1A6B2}

    % commands and environments needed by pandoc snippets
    % extracted from the output of `pandoc -s`
    \providecommand{\tightlist}{%
      \setlength{\itemsep}{0pt}\setlength{\parskip}{0pt}}
    \DefineVerbatimEnvironment{Highlighting}{Verbatim}{commandchars=\\\{\}}
    % Add ',fontsize=\small' for more characters per line
    \newenvironment{Shaded}{}{}
    \newcommand{\KeywordTok}[1]{\textcolor[rgb]{0.00,0.44,0.13}{\textbf{{#1}}}}
    \newcommand{\DataTypeTok}[1]{\textcolor[rgb]{0.56,0.13,0.00}{{#1}}}
    \newcommand{\DecValTok}[1]{\textcolor[rgb]{0.25,0.63,0.44}{{#1}}}
    \newcommand{\BaseNTok}[1]{\textcolor[rgb]{0.25,0.63,0.44}{{#1}}}
    \newcommand{\FloatTok}[1]{\textcolor[rgb]{0.25,0.63,0.44}{{#1}}}
    \newcommand{\CharTok}[1]{\textcolor[rgb]{0.25,0.44,0.63}{{#1}}}
    \newcommand{\StringTok}[1]{\textcolor[rgb]{0.25,0.44,0.63}{{#1}}}
    \newcommand{\CommentTok}[1]{\textcolor[rgb]{0.38,0.63,0.69}{\textit{{#1}}}}
    \newcommand{\OtherTok}[1]{\textcolor[rgb]{0.00,0.44,0.13}{{#1}}}
    \newcommand{\AlertTok}[1]{\textcolor[rgb]{1.00,0.00,0.00}{\textbf{{#1}}}}
    \newcommand{\FunctionTok}[1]{\textcolor[rgb]{0.02,0.16,0.49}{{#1}}}
    \newcommand{\RegionMarkerTok}[1]{{#1}}
    \newcommand{\ErrorTok}[1]{\textcolor[rgb]{1.00,0.00,0.00}{\textbf{{#1}}}}
    \newcommand{\NormalTok}[1]{{#1}}
    
    % Additional commands for more recent versions of Pandoc
    \newcommand{\ConstantTok}[1]{\textcolor[rgb]{0.53,0.00,0.00}{{#1}}}
    \newcommand{\SpecialCharTok}[1]{\textcolor[rgb]{0.25,0.44,0.63}{{#1}}}
    \newcommand{\VerbatimStringTok}[1]{\textcolor[rgb]{0.25,0.44,0.63}{{#1}}}
    \newcommand{\SpecialStringTok}[1]{\textcolor[rgb]{0.73,0.40,0.53}{{#1}}}
    \newcommand{\ImportTok}[1]{{#1}}
    \newcommand{\DocumentationTok}[1]{\textcolor[rgb]{0.73,0.13,0.13}{\textit{{#1}}}}
    \newcommand{\AnnotationTok}[1]{\textcolor[rgb]{0.38,0.63,0.69}{\textbf{\textit{{#1}}}}}
    \newcommand{\CommentVarTok}[1]{\textcolor[rgb]{0.38,0.63,0.69}{\textbf{\textit{{#1}}}}}
    \newcommand{\VariableTok}[1]{\textcolor[rgb]{0.10,0.09,0.49}{{#1}}}
    \newcommand{\ControlFlowTok}[1]{\textcolor[rgb]{0.00,0.44,0.13}{\textbf{{#1}}}}
    \newcommand{\OperatorTok}[1]{\textcolor[rgb]{0.40,0.40,0.40}{{#1}}}
    \newcommand{\BuiltInTok}[1]{{#1}}
    \newcommand{\ExtensionTok}[1]{{#1}}
    \newcommand{\PreprocessorTok}[1]{\textcolor[rgb]{0.74,0.48,0.00}{{#1}}}
    \newcommand{\AttributeTok}[1]{\textcolor[rgb]{0.49,0.56,0.16}{{#1}}}
    \newcommand{\InformationTok}[1]{\textcolor[rgb]{0.38,0.63,0.69}{\textbf{\textit{{#1}}}}}
    \newcommand{\WarningTok}[1]{\textcolor[rgb]{0.38,0.63,0.69}{\textbf{\textit{{#1}}}}}
    
    
    % Define a nice break command that doesn't care if a line doesn't already
    % exist.
    \def\br{\hspace*{\fill} \\* }
    % Math Jax compatability definitions
    \def\gt{>}
    \def\lt{<}
    % Document parameters
    \title{project2\_checkpoint-Copy1}
    
    
    

    % Pygments definitions
    
\makeatletter
\def\PY@reset{\let\PY@it=\relax \let\PY@bf=\relax%
    \let\PY@ul=\relax \let\PY@tc=\relax%
    \let\PY@bc=\relax \let\PY@ff=\relax}
\def\PY@tok#1{\csname PY@tok@#1\endcsname}
\def\PY@toks#1+{\ifx\relax#1\empty\else%
    \PY@tok{#1}\expandafter\PY@toks\fi}
\def\PY@do#1{\PY@bc{\PY@tc{\PY@ul{%
    \PY@it{\PY@bf{\PY@ff{#1}}}}}}}
\def\PY#1#2{\PY@reset\PY@toks#1+\relax+\PY@do{#2}}

\expandafter\def\csname PY@tok@w\endcsname{\def\PY@tc##1{\textcolor[rgb]{0.73,0.73,0.73}{##1}}}
\expandafter\def\csname PY@tok@c\endcsname{\let\PY@it=\textit\def\PY@tc##1{\textcolor[rgb]{0.25,0.50,0.50}{##1}}}
\expandafter\def\csname PY@tok@cp\endcsname{\def\PY@tc##1{\textcolor[rgb]{0.74,0.48,0.00}{##1}}}
\expandafter\def\csname PY@tok@k\endcsname{\let\PY@bf=\textbf\def\PY@tc##1{\textcolor[rgb]{0.00,0.50,0.00}{##1}}}
\expandafter\def\csname PY@tok@kp\endcsname{\def\PY@tc##1{\textcolor[rgb]{0.00,0.50,0.00}{##1}}}
\expandafter\def\csname PY@tok@kt\endcsname{\def\PY@tc##1{\textcolor[rgb]{0.69,0.00,0.25}{##1}}}
\expandafter\def\csname PY@tok@o\endcsname{\def\PY@tc##1{\textcolor[rgb]{0.40,0.40,0.40}{##1}}}
\expandafter\def\csname PY@tok@ow\endcsname{\let\PY@bf=\textbf\def\PY@tc##1{\textcolor[rgb]{0.67,0.13,1.00}{##1}}}
\expandafter\def\csname PY@tok@nb\endcsname{\def\PY@tc##1{\textcolor[rgb]{0.00,0.50,0.00}{##1}}}
\expandafter\def\csname PY@tok@nf\endcsname{\def\PY@tc##1{\textcolor[rgb]{0.00,0.00,1.00}{##1}}}
\expandafter\def\csname PY@tok@nc\endcsname{\let\PY@bf=\textbf\def\PY@tc##1{\textcolor[rgb]{0.00,0.00,1.00}{##1}}}
\expandafter\def\csname PY@tok@nn\endcsname{\let\PY@bf=\textbf\def\PY@tc##1{\textcolor[rgb]{0.00,0.00,1.00}{##1}}}
\expandafter\def\csname PY@tok@ne\endcsname{\let\PY@bf=\textbf\def\PY@tc##1{\textcolor[rgb]{0.82,0.25,0.23}{##1}}}
\expandafter\def\csname PY@tok@nv\endcsname{\def\PY@tc##1{\textcolor[rgb]{0.10,0.09,0.49}{##1}}}
\expandafter\def\csname PY@tok@no\endcsname{\def\PY@tc##1{\textcolor[rgb]{0.53,0.00,0.00}{##1}}}
\expandafter\def\csname PY@tok@nl\endcsname{\def\PY@tc##1{\textcolor[rgb]{0.63,0.63,0.00}{##1}}}
\expandafter\def\csname PY@tok@ni\endcsname{\let\PY@bf=\textbf\def\PY@tc##1{\textcolor[rgb]{0.60,0.60,0.60}{##1}}}
\expandafter\def\csname PY@tok@na\endcsname{\def\PY@tc##1{\textcolor[rgb]{0.49,0.56,0.16}{##1}}}
\expandafter\def\csname PY@tok@nt\endcsname{\let\PY@bf=\textbf\def\PY@tc##1{\textcolor[rgb]{0.00,0.50,0.00}{##1}}}
\expandafter\def\csname PY@tok@nd\endcsname{\def\PY@tc##1{\textcolor[rgb]{0.67,0.13,1.00}{##1}}}
\expandafter\def\csname PY@tok@s\endcsname{\def\PY@tc##1{\textcolor[rgb]{0.73,0.13,0.13}{##1}}}
\expandafter\def\csname PY@tok@sd\endcsname{\let\PY@it=\textit\def\PY@tc##1{\textcolor[rgb]{0.73,0.13,0.13}{##1}}}
\expandafter\def\csname PY@tok@si\endcsname{\let\PY@bf=\textbf\def\PY@tc##1{\textcolor[rgb]{0.73,0.40,0.53}{##1}}}
\expandafter\def\csname PY@tok@se\endcsname{\let\PY@bf=\textbf\def\PY@tc##1{\textcolor[rgb]{0.73,0.40,0.13}{##1}}}
\expandafter\def\csname PY@tok@sr\endcsname{\def\PY@tc##1{\textcolor[rgb]{0.73,0.40,0.53}{##1}}}
\expandafter\def\csname PY@tok@ss\endcsname{\def\PY@tc##1{\textcolor[rgb]{0.10,0.09,0.49}{##1}}}
\expandafter\def\csname PY@tok@sx\endcsname{\def\PY@tc##1{\textcolor[rgb]{0.00,0.50,0.00}{##1}}}
\expandafter\def\csname PY@tok@m\endcsname{\def\PY@tc##1{\textcolor[rgb]{0.40,0.40,0.40}{##1}}}
\expandafter\def\csname PY@tok@gh\endcsname{\let\PY@bf=\textbf\def\PY@tc##1{\textcolor[rgb]{0.00,0.00,0.50}{##1}}}
\expandafter\def\csname PY@tok@gu\endcsname{\let\PY@bf=\textbf\def\PY@tc##1{\textcolor[rgb]{0.50,0.00,0.50}{##1}}}
\expandafter\def\csname PY@tok@gd\endcsname{\def\PY@tc##1{\textcolor[rgb]{0.63,0.00,0.00}{##1}}}
\expandafter\def\csname PY@tok@gi\endcsname{\def\PY@tc##1{\textcolor[rgb]{0.00,0.63,0.00}{##1}}}
\expandafter\def\csname PY@tok@gr\endcsname{\def\PY@tc##1{\textcolor[rgb]{1.00,0.00,0.00}{##1}}}
\expandafter\def\csname PY@tok@ge\endcsname{\let\PY@it=\textit}
\expandafter\def\csname PY@tok@gs\endcsname{\let\PY@bf=\textbf}
\expandafter\def\csname PY@tok@gp\endcsname{\let\PY@bf=\textbf\def\PY@tc##1{\textcolor[rgb]{0.00,0.00,0.50}{##1}}}
\expandafter\def\csname PY@tok@go\endcsname{\def\PY@tc##1{\textcolor[rgb]{0.53,0.53,0.53}{##1}}}
\expandafter\def\csname PY@tok@gt\endcsname{\def\PY@tc##1{\textcolor[rgb]{0.00,0.27,0.87}{##1}}}
\expandafter\def\csname PY@tok@err\endcsname{\def\PY@bc##1{\setlength{\fboxsep}{0pt}\fcolorbox[rgb]{1.00,0.00,0.00}{1,1,1}{\strut ##1}}}
\expandafter\def\csname PY@tok@kc\endcsname{\let\PY@bf=\textbf\def\PY@tc##1{\textcolor[rgb]{0.00,0.50,0.00}{##1}}}
\expandafter\def\csname PY@tok@kd\endcsname{\let\PY@bf=\textbf\def\PY@tc##1{\textcolor[rgb]{0.00,0.50,0.00}{##1}}}
\expandafter\def\csname PY@tok@kn\endcsname{\let\PY@bf=\textbf\def\PY@tc##1{\textcolor[rgb]{0.00,0.50,0.00}{##1}}}
\expandafter\def\csname PY@tok@kr\endcsname{\let\PY@bf=\textbf\def\PY@tc##1{\textcolor[rgb]{0.00,0.50,0.00}{##1}}}
\expandafter\def\csname PY@tok@bp\endcsname{\def\PY@tc##1{\textcolor[rgb]{0.00,0.50,0.00}{##1}}}
\expandafter\def\csname PY@tok@fm\endcsname{\def\PY@tc##1{\textcolor[rgb]{0.00,0.00,1.00}{##1}}}
\expandafter\def\csname PY@tok@vc\endcsname{\def\PY@tc##1{\textcolor[rgb]{0.10,0.09,0.49}{##1}}}
\expandafter\def\csname PY@tok@vg\endcsname{\def\PY@tc##1{\textcolor[rgb]{0.10,0.09,0.49}{##1}}}
\expandafter\def\csname PY@tok@vi\endcsname{\def\PY@tc##1{\textcolor[rgb]{0.10,0.09,0.49}{##1}}}
\expandafter\def\csname PY@tok@vm\endcsname{\def\PY@tc##1{\textcolor[rgb]{0.10,0.09,0.49}{##1}}}
\expandafter\def\csname PY@tok@sa\endcsname{\def\PY@tc##1{\textcolor[rgb]{0.73,0.13,0.13}{##1}}}
\expandafter\def\csname PY@tok@sb\endcsname{\def\PY@tc##1{\textcolor[rgb]{0.73,0.13,0.13}{##1}}}
\expandafter\def\csname PY@tok@sc\endcsname{\def\PY@tc##1{\textcolor[rgb]{0.73,0.13,0.13}{##1}}}
\expandafter\def\csname PY@tok@dl\endcsname{\def\PY@tc##1{\textcolor[rgb]{0.73,0.13,0.13}{##1}}}
\expandafter\def\csname PY@tok@s2\endcsname{\def\PY@tc##1{\textcolor[rgb]{0.73,0.13,0.13}{##1}}}
\expandafter\def\csname PY@tok@sh\endcsname{\def\PY@tc##1{\textcolor[rgb]{0.73,0.13,0.13}{##1}}}
\expandafter\def\csname PY@tok@s1\endcsname{\def\PY@tc##1{\textcolor[rgb]{0.73,0.13,0.13}{##1}}}
\expandafter\def\csname PY@tok@mb\endcsname{\def\PY@tc##1{\textcolor[rgb]{0.40,0.40,0.40}{##1}}}
\expandafter\def\csname PY@tok@mf\endcsname{\def\PY@tc##1{\textcolor[rgb]{0.40,0.40,0.40}{##1}}}
\expandafter\def\csname PY@tok@mh\endcsname{\def\PY@tc##1{\textcolor[rgb]{0.40,0.40,0.40}{##1}}}
\expandafter\def\csname PY@tok@mi\endcsname{\def\PY@tc##1{\textcolor[rgb]{0.40,0.40,0.40}{##1}}}
\expandafter\def\csname PY@tok@il\endcsname{\def\PY@tc##1{\textcolor[rgb]{0.40,0.40,0.40}{##1}}}
\expandafter\def\csname PY@tok@mo\endcsname{\def\PY@tc##1{\textcolor[rgb]{0.40,0.40,0.40}{##1}}}
\expandafter\def\csname PY@tok@ch\endcsname{\let\PY@it=\textit\def\PY@tc##1{\textcolor[rgb]{0.25,0.50,0.50}{##1}}}
\expandafter\def\csname PY@tok@cm\endcsname{\let\PY@it=\textit\def\PY@tc##1{\textcolor[rgb]{0.25,0.50,0.50}{##1}}}
\expandafter\def\csname PY@tok@cpf\endcsname{\let\PY@it=\textit\def\PY@tc##1{\textcolor[rgb]{0.25,0.50,0.50}{##1}}}
\expandafter\def\csname PY@tok@c1\endcsname{\let\PY@it=\textit\def\PY@tc##1{\textcolor[rgb]{0.25,0.50,0.50}{##1}}}
\expandafter\def\csname PY@tok@cs\endcsname{\let\PY@it=\textit\def\PY@tc##1{\textcolor[rgb]{0.25,0.50,0.50}{##1}}}

\def\PYZbs{\char`\\}
\def\PYZus{\char`\_}
\def\PYZob{\char`\{}
\def\PYZcb{\char`\}}
\def\PYZca{\char`\^}
\def\PYZam{\char`\&}
\def\PYZlt{\char`\<}
\def\PYZgt{\char`\>}
\def\PYZsh{\char`\#}
\def\PYZpc{\char`\%}
\def\PYZdl{\char`\$}
\def\PYZhy{\char`\-}
\def\PYZsq{\char`\'}
\def\PYZdq{\char`\"}
\def\PYZti{\char`\~}
% for compatibility with earlier versions
\def\PYZat{@}
\def\PYZlb{[}
\def\PYZrb{]}
\makeatother


    % Exact colors from NB
    \definecolor{incolor}{rgb}{0.0, 0.0, 0.5}
    \definecolor{outcolor}{rgb}{0.545, 0.0, 0.0}



    
    % Prevent overflowing lines due to hard-to-break entities
    \sloppy 
    % Setup hyperref package
    \hypersetup{
      breaklinks=true,  % so long urls are correctly broken across lines
      colorlinks=true,
      urlcolor=urlcolor,
      linkcolor=linkcolor,
      citecolor=citecolor,
      }
    % Slightly bigger margins than the latex defaults
    
    \geometry{verbose,tmargin=1in,bmargin=1in,lmargin=1in,rmargin=1in}
    
    

    \begin{document}
    
    
    \maketitle
    
    

    
    \hypertarget{investigating-heat-transfer-as-a-function-of-volume}{%
\section{Investigating Heat Transfer as a Function of
Volume}\label{investigating-heat-transfer-as-a-function-of-volume}}

    \begin{Verbatim}[commandchars=\\\{\}]
{\color{incolor}In [{\color{incolor}1}]:} \PY{c+c1}{\PYZsh{} Configure Jupyter so figures appear in the notebook}
        \PY{o}{\PYZpc{}}\PY{k}{matplotlib} inline
        
        \PY{c+c1}{\PYZsh{} Configure Jupyter to display the assigned value after an assignment}
        \PY{o}{\PYZpc{}}\PY{k}{config} InteractiveShell.ast\PYZus{}node\PYZus{}interactivity=\PYZsq{}last\PYZus{}expr\PYZus{}or\PYZus{}assign\PYZsq{}
        
        \PY{c+c1}{\PYZsh{} import functions from the modsim.py module}
        \PY{k+kn}{from} \PY{n+nn}{modsim} \PY{k}{import} \PY{o}{*}
        \PY{k+kn}{import} \PY{n+nn}{pandas} \PY{k}{as} \PY{n+nn}{pd}
        \PY{k+kn}{from} \PY{n+nn}{scipy} \PY{k}{import} \PY{n}{stats}
        \PY{k+kn}{import} \PY{n+nn}{numpy} \PY{k}{as} \PY{n+nn}{np}
\end{Verbatim}


    \hypertarget{question}{%
\subsection{Question}\label{question}}

The phenomenon in question is a universal one - one that anyone who
consumes hot beverages experiences on a regular occurrence: As a hot
drink is being consumed, the liquid looses heat faster the less volume
there is left. The resulting experience is a colder drink the faster it
is consumed. Thus, we set out to answer the questions: 1. How can this
relationship between changing volume and changing rate of temperature
drop be qualitatively explained? 2. Can we predict the final temperature
of a fluid at a given time based on its starting temperature, the
temperature of its environment, the heat transfer coefficient specific
to its container, the dimensions of the container (a rectangular prism
with a square base), and its rate of change of volume (assumed to be
constant)?

\hypertarget{model}{%
\section{Model}\label{model}}

To answer these questions, we start with \(dTdt = -r * (T - T_e)\) where
\(T_e\) is the temperature of the environment, and \(r\) is a constant
that represents: \((h*A)/(C*m)\) where \(h\) is the heat transfer
coefficient, \(A\) is the surface area of convection, \(C\) is the
specific heat of water, and \(m\) is the mass of the water. Thus, \(r\)
can be represented as a function of the volume of water. Some
assumptions are made in our model: 1. The heat transfer is assumed to be
constant over all surfaces of the container, which in the physical world
isn't true: the heat transfer coefficient is different depending on
whether the water is in contact with the air or with the walls of the
container. Because the container we are modeling is a plastic container
and therefore of low conductivity, we believe this is negligible. 2. The
dimensions of the container: because the dimensions of the container
must be defined to avoid a two-dimensional function, we reference the
container used in the later experiment: a 0.13m x 0.13m x 0.05m plastic
container. 3. The heat transfer constant: because the heat transfer
constant is an empirically determined value, we chose the value of
\texttt{h\ =\ 0.09} to match our mathematical model as close as possible
to our empirical model (seen later in the essay). While this is an
empirically-driven mathematical model, these values must be measured to
achieve a two-dimensional result. Thus, our results are constrained to
interpretation only in the context of our specific container, but the
model is flexible enough to accommodate any container so long as the
empirical information is supplied.

The function that takes \texttt{V} as an argument and returns \(r\)
(\texttt{r}) is shown below:

    \begin{Verbatim}[commandchars=\\\{\}]
{\color{incolor}In [{\color{incolor}2}]:} \PY{c+c1}{\PYZsh{} Function of r using math:}
        \PY{k}{def} \PY{n+nf}{r\PYZus{}math}\PY{p}{(}\PY{n}{V}\PY{p}{)}\PY{p}{:}
            \PY{n}{V} \PY{o}{=} \PY{n+nb}{round}\PY{p}{(}\PY{n}{V}\PY{p}{,} \PY{l+m+mi}{2}\PY{p}{)}
            \PY{k}{if} \PY{n}{V} \PY{o}{\PYZgt{}} \PY{l+m+mi}{0}\PY{p}{:}
                \PY{c+c1}{\PYZsh{} size of box:}
                \PY{n}{x} \PY{o}{=} \PY{o}{.}\PY{l+m+mi}{13}
                
                \PY{n}{mass} \PY{o}{=} \PY{n}{V}\PY{o}{*}\PY{l+m+mf}{0.001}
        
                \PY{c+c1}{\PYZsh{} heat transfer coefficient between air and water}
                \PY{n}{h} \PY{o}{=} \PY{l+m+mf}{0.09}
                \PY{c+c1}{\PYZsh{} heat capacity of water}
                \PY{n}{C} \PY{o}{=} \PY{l+m+mf}{4.184}
        
                \PY{c+c1}{\PYZsh{} surface area as a function of volume (volume in the equation is in kiloliters, so it}
                \PY{c+c1}{\PYZsh{} needs to be converted to milliliters)}
                \PY{n}{SA} \PY{o}{=} \PY{l+m+mi}{2}\PY{o}{*}\PY{p}{(}\PY{n}{x}\PY{o}{*}\PY{o}{*}\PY{l+m+mi}{2}\PY{p}{)} \PY{o}{+} \PY{p}{(}\PY{l+m+mi}{4}\PY{o}{*}\PY{p}{(}\PY{n}{V}\PY{o}{*} \PY{l+m+mf}{0.000001}\PY{p}{)}\PY{p}{)}\PY{o}{/}\PY{n}{x}
            
            \PY{k}{else}\PY{p}{:}
                \PY{n}{x} \PY{o}{=} \PY{o}{.}\PY{l+m+mi}{13}
                \PY{n}{mass} \PY{o}{=} \PY{o}{.}\PY{l+m+mi}{001}
                \PY{n}{h} \PY{o}{=} \PY{l+m+mf}{0.09}
                \PY{n}{C} \PY{o}{=} \PY{l+m+mf}{4.184}
                \PY{n}{SA} \PY{o}{=} \PY{l+m+mi}{2} \PY{o}{*} \PY{p}{(}\PY{n}{x}\PY{o}{*}\PY{o}{*}\PY{l+m+mi}{2}\PY{p}{)} \PY{o}{+} \PY{p}{(}\PY{l+m+mi}{4} \PY{o}{*} \PY{p}{(}\PY{n}{V} \PY{o}{*} \PY{l+m+mf}{0.000001}\PY{p}{)}\PY{p}{)} \PY{o}{/} \PY{n}{x}
            \PY{k}{return} \PY{p}{(}\PY{p}{(}\PY{n}{h}\PY{o}{*}\PY{n}{SA}\PY{p}{)}\PY{o}{/}\PY{p}{(}\PY{n}{C}\PY{o}{*}\PY{n}{mass}\PY{p}{)}\PY{p}{)}
\end{Verbatim}


    \begin{Verbatim}[commandchars=\\\{\}]
{\color{incolor}In [{\color{incolor}3}]:} \PY{c+c1}{\PYZsh{} Visualizing r\PYZus{}math:}
        \PY{n}{listv} \PY{o}{=} \PY{p}{[}\PY{p}{]}
        \PY{n}{listr} \PY{o}{=} \PY{p}{[}\PY{p}{]}
        \PY{k}{for} \PY{n}{V} \PY{o+ow}{in} \PY{n+nb}{range}\PY{p}{(}\PY{l+m+mi}{20}\PY{p}{,} \PY{l+m+mi}{700}\PY{p}{)}\PY{p}{:}
            \PY{n}{listv}\PY{o}{.}\PY{n}{append}\PY{p}{(}\PY{n}{V}\PY{p}{)}
            \PY{n}{listr}\PY{o}{.}\PY{n}{append}\PY{p}{(}\PY{n}{r\PYZus{}math}\PY{p}{(}\PY{n}{V}\PY{p}{)}\PY{p}{)}
            
        \PY{n}{plot}\PY{p}{(}\PY{n}{listv}\PY{p}{,} \PY{n}{listr}\PY{p}{)}
        \PY{n}{decorate}\PY{p}{(}\PY{n}{xlabel} \PY{o}{=} \PY{l+s+s1}{\PYZsq{}}\PY{l+s+s1}{Volume (mL)}\PY{l+s+s1}{\PYZsq{}}\PY{p}{,}
                 \PY{n}{ylabel} \PY{o}{=} \PY{l+s+s1}{\PYZsq{}}\PY{l+s+s1}{r}\PY{l+s+s1}{\PYZsq{}}\PY{p}{,}
                 \PY{n}{title} \PY{o}{=} \PY{l+s+s1}{\PYZsq{}}\PY{l+s+s1}{Mathematical Function of r vs Volume}\PY{l+s+s1}{\PYZsq{}}\PY{p}{)}
\end{Verbatim}


    \begin{center}
    \adjustimage{max size={0.9\linewidth}{0.9\paperheight}}{output_4_0.png}
    \end{center}
    { \hspace*{\fill} \\}
    
    When using dTdt to update the temperature stock of the water, instead of
passing in \(r\) as a constant, it is passed in as \texttt{r\_math}
evaluated at the volume (\texttt{V}) at the given time step. The volume
too doesn't remain constant because it is being swept; the volume is
passed in by \texttt{dvdt}.

    \begin{Verbatim}[commandchars=\\\{\}]
{\color{incolor}In [{\color{incolor}4}]:} \PY{k}{def} \PY{n+nf}{update\PYZus{}func}\PY{p}{(}\PY{n}{state}\PY{p}{,} \PY{n}{t}\PY{p}{,} \PY{n}{system}\PY{p}{,} \PY{n}{dvdt}\PY{p}{,} \PY{n}{r\PYZus{}math}\PY{p}{)}\PY{p}{:}
            \PY{l+s+sd}{\PYZdq{}\PYZdq{}\PYZdq{}Update the thermal transfer model.}
        \PY{l+s+sd}{    }
        \PY{l+s+sd}{    State objects:}
        \PY{l+s+sd}{    Temperature of vessle}
        \PY{l+s+sd}{    Volume of vessle}
        \PY{l+s+sd}{    }
        \PY{l+s+sd}{    t \PYZhy{}\PYZhy{}\PYZgt{} time}
        \PY{l+s+sd}{    T \PYZhy{}\PYZhy{}\PYZgt{} Temperature}
        \PY{l+s+sd}{    }
        \PY{l+s+sd}{    returns: State object containing the stocks}
        \PY{l+s+sd}{    \PYZdq{}\PYZdq{}\PYZdq{}}
            \PY{n}{unpack}\PY{p}{(}\PY{n}{system}\PY{p}{)}
            
            \PY{n}{T} \PY{o}{=} \PY{n}{state}\PY{o}{.}\PY{n}{T}
            \PY{n}{V} \PY{o}{=} \PY{n}{state}\PY{o}{.}\PY{n}{V}
            
            \PY{n}{r} \PY{o}{=} \PY{n}{r\PYZus{}math}\PY{p}{(}\PY{n}{state}\PY{o}{.}\PY{n}{V}\PY{p}{)}
                
            \PY{n}{dTdt} \PY{o}{=} \PY{o}{\PYZhy{}}\PY{n}{r} \PY{o}{*} \PY{p}{(}\PY{n}{T} \PY{o}{\PYZhy{}} \PY{n}{T\PYZus{}env}\PY{p}{)}
            \PY{c+c1}{\PYZsh{} dvdt is passed in as an argument \PYZhy{} it is being swept}
        
            \PY{k}{if}\PY{p}{(}\PY{n}{V}\PY{o}{\PYZgt{}}\PY{l+m+mi}{0} \PY{o+ow}{and} \PY{n}{T} \PY{o}{\PYZgt{}}\PY{l+m+mi}{25}\PY{p}{)}\PY{p}{:}
                \PY{n}{T} \PY{o}{+}\PY{o}{=} \PY{n}{dTdt} \PY{o}{*} \PY{n}{dt}
                \PY{n}{V} \PY{o}{+}\PY{o}{=} \PY{n}{dvdt} \PY{o}{*} \PY{n}{dt}
            \PY{k}{else}\PY{p}{:}
                \PY{n}{V} \PY{o}{=} \PY{l+m+mi}{0}
                \PY{n}{T} \PY{o}{=} \PY{k+kc}{None}
            \PY{k}{return} \PY{n}{State}\PY{p}{(}\PY{n}{T}\PY{o}{=}\PY{n}{T}\PY{p}{,} \PY{n}{V}\PY{o}{=}\PY{n}{V}\PY{p}{)}
\end{Verbatim}


    \begin{Verbatim}[commandchars=\\\{\}]
{\color{incolor}In [{\color{incolor}5}]:} \PY{k}{def} \PY{n+nf}{run\PYZus{}simulation}\PY{p}{(}\PY{n}{system}\PY{p}{,} \PY{n}{update\PYZus{}func}\PY{p}{,} \PY{n}{dvdt}\PY{p}{,} \PY{n}{r\PYZus{}math}\PY{p}{)}\PY{p}{:}
            \PY{l+s+sd}{\PYZdq{}\PYZdq{}\PYZdq{}Runs a simulation of the system.}
        \PY{l+s+sd}{    }
        \PY{l+s+sd}{    Add a TimeFrame to the System: results}
        \PY{l+s+sd}{    \PYZdq{}\PYZdq{}\PYZdq{}}
            \PY{n}{unpack}\PY{p}{(}\PY{n}{system}\PY{p}{)}
            
            \PY{n}{frame} \PY{o}{=} \PY{n}{TimeFrame}\PY{p}{(}\PY{n}{columns}\PY{o}{=}\PY{n}{init}\PY{o}{.}\PY{n}{index}\PY{p}{)}
            \PY{n}{frame}\PY{o}{.}\PY{n}{row}\PY{p}{[}\PY{l+m+mi}{0}\PY{p}{]} \PY{o}{=} \PY{n}{init}
            \PY{n}{ts} \PY{o}{=} \PY{n}{linrange}\PY{p}{(}\PY{l+m+mi}{0}\PY{p}{,} \PY{n}{t\PYZus{}end}\PY{p}{,} \PY{n}{dt}\PY{p}{)}
            
            \PY{k}{for} \PY{n}{t} \PY{o+ow}{in} \PY{n}{ts}\PY{p}{:}
                \PY{n}{frame}\PY{o}{.}\PY{n}{row}\PY{p}{[}\PY{n}{t}\PY{o}{+}\PY{n}{dt}\PY{p}{]} \PY{o}{=} \PY{n}{update\PYZus{}func}\PY{p}{(}\PY{n}{frame}\PY{o}{.}\PY{n}{row}\PY{p}{[}\PY{n}{t}\PY{p}{]}\PY{p}{,} \PY{n}{t}\PY{p}{,} \PY{n}{system}\PY{p}{,} \PY{n}{dvdt}\PY{p}{,} \PY{n}{r\PYZus{}math}\PY{p}{)}
                
            \PY{c+c1}{\PYZsh{} store the final temperature in T\PYZus{}final}
            \PY{n}{system}\PY{o}{.}\PY{n}{T\PYZus{}final} \PY{o}{=} \PY{n}{get\PYZus{}last\PYZus{}value}\PY{p}{(}\PY{n}{frame}\PY{o}{.}\PY{n}{T}\PY{p}{)}
            
            \PY{k}{return} \PY{n}{frame}
\end{Verbatim}


    \begin{Verbatim}[commandchars=\\\{\}]
{\color{incolor}In [{\color{incolor}6}]:} \PY{k}{def} \PY{n+nf}{make\PYZus{}system}\PY{p}{(}\PY{n}{T\PYZus{}init}\PY{p}{,} \PY{n}{V\PYZus{}init}\PY{p}{,} \PY{n}{dvdt}\PY{p}{)}\PY{p}{:}
            \PY{l+s+sd}{\PYZdq{}\PYZdq{}\PYZdq{}Makes a System object with the given parameters.}
        
        \PY{l+s+sd}{    T\PYZus{}init: initial temperature in degC}
        \PY{l+s+sd}{    r: heat transfer rate \PYZhy{} when run, a function of r will be substituted in for it}
        \PY{l+s+sd}{    t\PYZus{}end: end time of simulation}
        \PY{l+s+sd}{    T\PYZus{}env: temperature of the environment}
        \PY{l+s+sd}{    dt: time step}
        \PY{l+s+sd}{    }
        \PY{l+s+sd}{    returns: System object}
        \PY{l+s+sd}{    \PYZdq{}\PYZdq{}\PYZdq{}}
            \PY{n}{init} \PY{o}{=} \PY{n}{State}\PY{p}{(}\PY{n}{T}\PY{o}{=}\PY{n}{T\PYZus{}init}\PY{p}{,} \PY{n}{V}\PY{o}{=}\PY{n}{V\PYZus{}init}\PY{p}{)}
            
            \PY{c+c1}{\PYZsh{} T\PYZus{}final is used to store the final temperature.}
            \PY{c+c1}{\PYZsh{} Before the simulation runs, T\PYZus{}final = T\PYZus{}init}
            \PY{c+c1}{\PYZsh{} We do the same thing for Volume}
            \PY{n}{T\PYZus{}final} \PY{o}{=} \PY{n}{T\PYZus{}init}
            \PY{n}{V\PYZus{}final} \PY{o}{=} \PY{n}{V\PYZus{}init}
            
            \PY{n}{T\PYZus{}env} \PY{o}{=} \PY{l+m+mi}{25}
            \PY{n}{dt} \PY{o}{=} \PY{l+m+mi}{1}
            
            \PY{c+c1}{\PYZsh{} n\PYZus{}sweep stands for number of sweeps that the model will perform}
            \PY{n}{n\PYZus{}sweep} \PY{o}{=} \PY{l+m+mi}{9}
            
            \PY{c+c1}{\PYZsh{} n\PYZus{}sec stands for number of seconds that the simulation will be run for}
            \PY{n}{t\PYZus{}end} \PY{o}{=} \PY{l+m+mi}{600}
            
            \PY{c+c1}{\PYZsh{} Used at the end of this simulation, this is the threshold that determines}
            \PY{c+c1}{\PYZsh{} what rates of volume change over a given time period will result in a }
            \PY{c+c1}{\PYZsh{} temperature at or over 40 degrees celcius.}
            \PY{n}{minimum\PYZus{}acceptable\PYZus{}temp} \PY{o}{=} \PY{l+m+mi}{40}
            
            \PY{k}{return} \PY{n}{System}\PY{p}{(}\PY{n+nb}{locals}\PY{p}{(}\PY{p}{)}\PY{p}{)}
        
        \PY{n}{system} \PY{o}{=} \PY{n}{make\PYZus{}system}\PY{p}{(}\PY{l+m+mi}{85}\PY{p}{,} \PY{l+m+mi}{700}\PY{p}{,} \PY{o}{\PYZhy{}}\PY{l+m+mi}{1}\PY{p}{)}\PY{p}{;}
\end{Verbatim}


    \begin{Verbatim}[commandchars=\\\{\}]
{\color{incolor}In [{\color{incolor}7}]:} \PY{k}{def} \PY{n+nf}{sweepdVdt}\PY{p}{(}\PY{n}{system}\PY{p}{)}\PY{p}{:}
            \PY{n}{sweep} \PY{o}{=} \PY{n}{SweepSeries}\PY{p}{(}\PY{p}{)}
            \PY{n}{dvdt\PYZus{}range} \PY{o}{=} \PY{n}{linspace}\PY{p}{(}\PY{o}{\PYZhy{}}\PY{l+m+mi}{5}\PY{p}{,}\PY{l+m+mi}{0}\PY{p}{,}\PY{n}{system}\PY{o}{.}\PY{n}{n\PYZus{}sweep}\PY{p}{)}
            
            \PY{k}{for} \PY{n}{t} \PY{o+ow}{in} \PY{n+nb}{range}\PY{p}{(}\PY{l+m+mi}{0}\PY{p}{,} \PY{n}{system}\PY{o}{.}\PY{n}{n\PYZus{}sweep}\PY{p}{)}\PY{p}{:}
                \PY{n}{dvdt} \PY{o}{=} \PY{n}{dvdt\PYZus{}range}\PY{p}{[}\PY{n}{t}\PY{p}{]}
                \PY{n}{sweep}\PY{p}{[}\PY{n}{t}\PY{p}{]} \PY{o}{=} \PY{n}{run\PYZus{}simulation}\PY{p}{(}\PY{n}{system}\PY{p}{,} \PY{n}{update\PYZus{}func}\PY{p}{,} \PY{n}{dvdt}\PY{p}{,} \PY{n}{r\PYZus{}math}\PY{p}{)}
        
            \PY{k}{return} \PY{n}{sweep}
        
        \PY{n}{swept} \PY{o}{=} \PY{n}{sweepdVdt}\PY{p}{(}\PY{n}{system}\PY{p}{)}\PY{p}{;}
\end{Verbatim}


    \hypertarget{qualitative-result-of-mathematical-model}{%
\subsubsection{Qualitative Result of Mathematical
Model:}\label{qualitative-result-of-mathematical-model}}

    \begin{Verbatim}[commandchars=\\\{\}]
{\color{incolor}In [{\color{incolor}8}]:} \PY{n}{dvdt\PYZus{}range} \PY{o}{=} \PY{n}{linspace}\PY{p}{(}\PY{o}{\PYZhy{}}\PY{l+m+mi}{5}\PY{p}{,}\PY{l+m+mi}{0}\PY{p}{,}\PY{n}{system}\PY{o}{.}\PY{n}{n\PYZus{}sweep}\PY{p}{)}
        \PY{n}{fig} \PY{o}{=} \PY{n}{plt}\PY{o}{.}\PY{n}{figure}\PY{p}{(}\PY{p}{)}
        \PY{n}{ax} \PY{o}{=} \PY{n}{plt}\PY{o}{.}\PY{n}{subplot}\PY{p}{(}\PY{l+m+mi}{111}\PY{p}{)}\PY{p}{;}
        \PY{k}{for} \PY{n}{i} \PY{o+ow}{in} \PY{n+nb}{range}\PY{p}{(}\PY{n}{system}\PY{o}{.}\PY{n}{n\PYZus{}sweep}\PY{p}{)}\PY{p}{:}
            \PY{n}{dvdt} \PY{o}{=} \PY{n+nb}{round}\PY{p}{(}\PY{n}{dvdt\PYZus{}range}\PY{p}{[}\PY{n}{i}\PY{p}{]}\PY{p}{,} \PY{l+m+mi}{2}\PY{p}{)}
            \PY{n}{ax}\PY{o}{.}\PY{n}{plot}\PY{p}{(}\PY{n}{swept}\PY{p}{[}\PY{n}{i}\PY{p}{]}\PY{o}{.}\PY{n}{index}\PY{p}{,} \PY{n}{swept}\PY{p}{[}\PY{n}{i}\PY{p}{]}\PY{o}{.}\PY{n}{T}\PY{p}{,} \PY{n}{label} \PY{o}{=} \PY{n+nb}{str}\PY{p}{(}\PY{n}{dvdt}\PY{p}{)} \PY{o}{+} \PY{l+s+s1}{\PYZsq{}}\PY{l+s+s1}{ mL/sec}\PY{l+s+s1}{\PYZsq{}}\PY{p}{)}
            \PY{n}{decorate}\PY{p}{(}\PY{n}{xlabel} \PY{o}{=} \PY{l+s+s1}{\PYZsq{}}\PY{l+s+s1}{Time (s)}\PY{l+s+s1}{\PYZsq{}}\PY{p}{,}
                    \PY{n}{ylabel} \PY{o}{=} \PY{l+s+s1}{\PYZsq{}}\PY{l+s+s1}{Temperature (degrees C)}\PY{l+s+s1}{\PYZsq{}}\PY{p}{,}
                    \PY{n}{title} \PY{o}{=} \PY{l+s+s1}{\PYZsq{}}\PY{l+s+s1}{Temperature of Coffee as Volume Decreases over }\PY{l+s+si}{\PYZob{}\PYZcb{}}\PY{l+s+s1}{ seconds}\PY{l+s+s1}{\PYZsq{}}\PY{o}{.}\PY{n}{format}\PY{p}{(}\PY{n}{system}\PY{o}{.}\PY{n}{t\PYZus{}end}\PY{p}{)}\PY{p}{)}
            \PY{n}{ax}\PY{o}{.}\PY{n}{legend}\PY{p}{(}\PY{n}{bbox\PYZus{}to\PYZus{}anchor}\PY{o}{=}\PY{p}{(}\PY{l+m+mf}{1.1}\PY{p}{,}\PY{l+m+mf}{1.05}\PY{p}{)}\PY{p}{)}
\end{Verbatim}


    \begin{center}
    \adjustimage{max size={0.9\linewidth}{0.9\paperheight}}{output_11_0.png}
    \end{center}
    { \hspace*{\fill} \\}
    
    The qualitative result appears to validate our experiences with
consuming hot liquids: the higher the rate of change of volume, the
quicker the drop in temperature. However, because this is a mathematical
model, when the volume is infinitely small, an infinite slope of
temperature over time results. This is not something that is observed in
the physical world, and is a shortcoming of the mathematical model.

    \hypertarget{quantitative-result-of-mathematical-model}{%
\subsubsection{Quantitative Result of Mathematical
Model}\label{quantitative-result-of-mathematical-model}}

The final temperatures of each \texttt{dvdt} sweep are plotted below,
along with a ``Minimum Acceptable Temperature.'' The following code
exists to extract those final temperatures, and the ``Minimum Acceptable
Temperature'' represents an arbitrary threshold that can be set to
determine what rates of change of volume will yield a desired
temperature. It is important to note that the following results are
specific to not only the container used for the experiment, but also the
600 second time frame that is modeled.

    \begin{Verbatim}[commandchars=\\\{\}]
{\color{incolor}In [{\color{incolor}9}]:} \PY{c+c1}{\PYZsh{} Creating the time series in which the last temperatures from each dVdt sweep will be stored.}
        \PY{n}{last\PYZus{}temp} \PY{o}{=} \PY{n}{TimeSeries}\PY{p}{(}\PY{p}{)}
        
        \PY{k}{def} \PY{n+nf}{determine\PYZus{}last\PYZus{}temp}\PY{p}{(}\PY{n}{last\PYZus{}temp}\PY{p}{,} \PY{n}{system}\PY{p}{,} \PY{n}{swept}\PY{p}{)}\PY{p}{:}
            \PY{c+c1}{\PYZsh{} Selecting the last values from each sweep by selecting the temperature associated with the point }
            \PY{c+c1}{\PYZsh{} at which the volume first hits 0 mL. If the volume never reaches 0, the last temperature is still selected. }
            \PY{k}{for} \PY{n}{i} \PY{o+ow}{in} \PY{n+nb}{range}\PY{p}{(}\PY{n}{system}\PY{o}{.}\PY{n}{n\PYZus{}sweep}\PY{p}{)}\PY{p}{:}
                \PY{k}{for} \PY{n}{t} \PY{o+ow}{in} \PY{n+nb}{range}\PY{p}{(}\PY{n}{system}\PY{o}{.}\PY{n}{t\PYZus{}end}\PY{p}{)}\PY{p}{:}
                    \PY{k}{if} \PY{p}{(}\PY{n}{swept}\PY{p}{[}\PY{n}{i}\PY{p}{]}\PY{o}{.}\PY{n}{V}\PY{p}{[}\PY{n}{t}\PY{p}{]} \PY{o}{==} \PY{l+m+mi}{0} \PY{o+ow}{or} \PY{n}{t} \PY{o}{==} \PY{n}{system}\PY{o}{.}\PY{n}{t\PYZus{}end} \PY{o}{\PYZhy{}} \PY{l+m+mi}{1}\PY{p}{)}\PY{p}{:}
                        \PY{n}{last\PYZus{}temp}\PY{p}{[}\PY{n}{i}\PY{p}{]} \PY{o}{=} \PY{n}{swept}\PY{p}{[}\PY{n}{i}\PY{p}{]}\PY{o}{.}\PY{n}{T}\PY{p}{[}\PY{n}{t}\PY{o}{\PYZhy{}}\PY{l+m+mi}{1}\PY{p}{]}
                        \PY{k}{break}
            \PY{k}{return} \PY{n}{last\PYZus{}temp}
        
        \PY{n}{last\PYZus{}temp} \PY{o}{=} \PY{n}{determine\PYZus{}last\PYZus{}temp}\PY{p}{(}\PY{n}{last\PYZus{}temp}\PY{p}{,} \PY{n}{system}\PY{p}{,} \PY{n}{swept}\PY{p}{)}
        
        \PY{c+c1}{\PYZsh{} To plot the data against dVdt and not against the sweep numbers, the dvdt values and last temperature}
        \PY{c+c1}{\PYZsh{} values will be converted into lists and then plotted against each other.}
        \PY{n}{dvdt\PYZus{}list} \PY{o}{=} \PY{p}{[}\PY{p}{]}
        \PY{n}{last\PYZus{}temp\PYZus{}list} \PY{o}{=} \PY{p}{[}\PY{p}{]}
        
        \PY{k}{for} \PY{n}{i} \PY{o+ow}{in} \PY{n+nb}{range}\PY{p}{(}\PY{n+nb}{len}\PY{p}{(}\PY{n}{dvdt\PYZus{}range}\PY{p}{)}\PY{p}{)}\PY{p}{:}
            \PY{n}{dvdt\PYZus{}list}\PY{o}{.}\PY{n}{append}\PY{p}{(}\PY{n}{dvdt\PYZus{}range}\PY{p}{[}\PY{n}{i}\PY{p}{]}\PY{p}{)}
            \PY{n}{last\PYZus{}temp\PYZus{}list}\PY{o}{.}\PY{n}{append}\PY{p}{(}\PY{n}{last\PYZus{}temp}\PY{p}{[}\PY{n}{i}\PY{p}{]}\PY{p}{)}
        
        \PY{k}{def} \PY{n+nf}{plot\PYZus{}final\PYZus{}temp\PYZus{}and\PYZus{}minimum\PYZus{}acceptable\PYZus{}temp}\PY{p}{(}\PY{n}{dvdt\PYZus{}list}\PY{p}{,} \PY{n}{last\PYZus{}temp\PYZus{}list}\PY{p}{,} \PY{n}{minimum\PYZus{}acceptable\PYZus{}temp}\PY{p}{)}\PY{p}{:}
            \PY{n}{plot}\PY{p}{(}\PY{n}{dvdt\PYZus{}list}\PY{p}{,} \PY{n}{last\PYZus{}temp\PYZus{}list}\PY{p}{,} \PY{n}{label} \PY{o}{=} \PY{l+s+s1}{\PYZsq{}}\PY{l+s+s1}{Final Temperature of Coffee}\PY{l+s+s1}{\PYZsq{}}\PY{p}{)}
            
            \PY{c+c1}{\PYZsh{} The minimum acceptable temperature will be plotted as a line. To achieve this, a list of length }
            \PY{c+c1}{\PYZsh{} len(dvdt\PYZus{}range) will created and contain the minimum acceptable temperature in each index. }
            \PY{n}{minimum\PYZus{}acceptable\PYZus{}temp\PYZus{}list} \PY{o}{=} \PY{n+nb}{list}\PY{p}{(}\PY{p}{)}
            \PY{k}{for} \PY{n}{i} \PY{o+ow}{in} \PY{n+nb}{range}\PY{p}{(}\PY{n+nb}{len}\PY{p}{(}\PY{n}{dvdt\PYZus{}range}\PY{p}{)}\PY{p}{)}\PY{p}{:}
                \PY{n}{minimum\PYZus{}acceptable\PYZus{}temp\PYZus{}list}\PY{o}{.}\PY{n}{append}\PY{p}{(}\PY{n}{minimum\PYZus{}acceptable\PYZus{}temp}\PY{p}{)}
            \PY{n}{plot}\PY{p}{(}\PY{n}{dvdt\PYZus{}list}\PY{p}{,} \PY{n}{minimum\PYZus{}acceptable\PYZus{}temp\PYZus{}list}\PY{p}{,} \PY{n}{label} \PY{o}{=} \PY{l+s+s1}{\PYZsq{}}\PY{l+s+s1}{Minimum Acceptable Temperature}\PY{l+s+s1}{\PYZsq{}}\PY{p}{)}
            \PY{k}{return} \PY{n}{minimum\PYZus{}acceptable\PYZus{}temp\PYZus{}list}
        
        \PY{n}{plot\PYZus{}final\PYZus{}temp\PYZus{}and\PYZus{}minimum\PYZus{}acceptable\PYZus{}temp} \PY{o}{=} \PY{n}{plot\PYZus{}final\PYZus{}temp\PYZus{}and\PYZus{}minimum\PYZus{}acceptable\PYZus{}temp}\PY{p}{(}\PY{n}{dvdt\PYZus{}list}\PY{p}{,} \PY{n}{last\PYZus{}temp\PYZus{}list}\PY{p}{,} 
                                                                                                  \PY{n}{minimum\PYZus{}acceptable\PYZus{}temp}\PY{p}{)}
        \PY{n}{decorate}\PY{p}{(}\PY{n}{xlabel} \PY{o}{=} \PY{l+s+s1}{\PYZsq{}}\PY{l+s+s1}{dVdt (mL/sec)}\PY{l+s+s1}{\PYZsq{}}\PY{p}{,} 
                 \PY{n}{ylabel} \PY{o}{=} \PY{l+s+s1}{\PYZsq{}}\PY{l+s+s1}{Final Temperature (degrees C)}\PY{l+s+s1}{\PYZsq{}}\PY{p}{,} 
                 \PY{n}{title} \PY{o}{=} \PY{l+s+s1}{\PYZsq{}}\PY{l+s+s1}{Final temperature of coffee after }\PY{l+s+si}{\PYZob{}\PYZcb{}}\PY{l+s+s1}{ seconds}\PY{l+s+s1}{\PYZsq{}}\PY{o}{.}\PY{n}{format}\PY{p}{(}\PY{n}{system}\PY{o}{.}\PY{n}{t\PYZus{}end}\PY{p}{)}\PY{p}{)}
\end{Verbatim}


    \begin{center}
    \adjustimage{max size={0.9\linewidth}{0.9\paperheight}}{output_14_0.png}
    \end{center}
    { \hspace*{\fill} \\}
    
    This graph illustrates that, given enough parameters/information about a
specific container, the final temperature can be predicted based on the
rate of change of volume of the container. A minimum is expected to
occur approximately dVdt = -2 because that indicates a cusp at which the
container will either have zero or nonzero volume remaining (to the
right and left of the cusp respectively). However, the graph is likely
inaccurate because it exhibits an odd behavior: a local minimum and
maximum occur between -4 \textless{} dVdt \textless{} -3. There is no
reason to suggest that this phenomenon should occur, as the relationship
between volume and r does not exhibit any inflection points. A likely
explanation for this shortcoming are the infinite slopes that exist in
the temperature vs.~time graphs when volume approaches 0, which may
result in inaccurate readings at the extremities of the graphs. This
mathematical model will be further analyzed in comparison to the
following empirical model.

    \hypertarget{validation-of-mathematical-model---repeating-model-using-empirical-data}{%
\section{Validation of Mathematical Model - Repeating Model using
Empirical
Data}\label{validation-of-mathematical-model---repeating-model-using-empirical-data}}

    In order to empirically obtain a graph of the heat transfer coefficient
as a function of volume, we obtained a series of empirically determined
r values (specific to the container used in the experiments) at
different volumes. Eleven different volumes in total were used, and to
calculate the r value, the following procedure was used: 1. Set up
experiment: connect a thermistor to a breadboard such that it is the
second resistor in the voltage divider. The voltage divider must be
connected to a 5V power supply and ground, and the first resistor must
have a resistance of 1k-ohm. Measure the voltage drop across the voltage
divider.\\
2. Bring water to near-boiling temperature. 3. Pour water into container
on a scale until the desired mass/volume of water is achieved 4. Insert
the thermistor into the water. 5. Record the starting voltage, as well
as the voltage after 5 minutes. 6. Convert the voltage to temperature
using the following equation: T = (ln(5/Voltage - 1)/3528 + 1/298)\^{}-1
- 273 7. Calculate an r value using the following equation: r = 1/t\_end
* log((T\_init - T\_env)/(T\_end - T\_env)) 8. Repeat steps 2-7 for as
many volumes are desired.

    \begin{Verbatim}[commandchars=\\\{\}]
{\color{incolor}In [{\color{incolor}10}]:} \PY{c+c1}{\PYZsh{} A list of the data collected in our experiments. Each volume has a corresponding calculated r value}
         \PY{n}{data} \PY{o}{=} \PY{p}{\PYZob{}}\PY{l+s+s1}{\PYZsq{}}\PY{l+s+s1}{Volume}\PY{l+s+s1}{\PYZsq{}}\PY{p}{:} \PY{p}{[}\PY{l+m+mf}{118.354}\PY{p}{,} \PY{l+m+mf}{169.507}\PY{p}{,}\PY{l+m+mf}{202.606}\PY{p}{,}\PY{l+m+mf}{258.774}\PY{p}{,}\PY{l+m+mf}{316.948}\PY{p}{,}\PY{l+m+mf}{377.128}\PY{p}{,}\PY{l+m+mf}{401.2}\PY{p}{,}\PY{l+m+mf}{458.371}\PY{p}{,}\PY{l+m+mf}{551.65}\PY{p}{,}\PY{l+m+mf}{654.959}\PY{p}{,}\PY{l+m+mf}{714.136}\PY{p}{,}\PY{p}{]}\PY{p}{,} 
                      \PY{l+s+s1}{\PYZsq{}}\PY{l+s+s1}{r}\PY{l+s+s1}{\PYZsq{}}\PY{p}{:} \PY{p}{[}\PY{l+m+mf}{0.007660215}\PY{p}{,}\PY{l+m+mf}{0.00563342}\PY{p}{,}\PY{l+m+mf}{0.004614561}\PY{p}{,}\PY{l+m+mf}{0.003912183}\PY{p}{,}\PY{l+m+mf}{0.003744291}\PY{p}{,}\PY{l+m+mf}{0.003259004}\PY{p}{,}\PY{l+m+mf}{0.002769218}\PY{p}{,}
                            \PY{l+m+mf}{0.002814454}\PY{p}{,}\PY{l+m+mf}{0.002157224}\PY{p}{,}\PY{l+m+mf}{0.001998175}\PY{p}{,}\PY{l+m+mf}{0.001906878}\PY{p}{]}\PY{p}{\PYZcb{}}
         \PY{c+c1}{\PYZsh{} Creating a DataFrame from our collected data:}
         \PY{n}{r\PYZus{}vs\PYZus{}volume\PYZus{}data} \PY{o}{=} \PY{n}{pd}\PY{o}{.}\PY{n}{DataFrame}\PY{p}{(}\PY{n}{data}\PY{o}{=}\PY{n}{data}\PY{p}{)}\PY{p}{;}
         
         \PY{n}{plot}\PY{p}{(}\PY{n}{r\PYZus{}vs\PYZus{}volume\PYZus{}data}\PY{p}{[}\PY{l+s+s1}{\PYZsq{}}\PY{l+s+s1}{Volume}\PY{l+s+s1}{\PYZsq{}}\PY{p}{]}\PY{p}{,} \PY{n}{r\PYZus{}vs\PYZus{}volume\PYZus{}data}\PY{p}{[}\PY{l+s+s1}{\PYZsq{}}\PY{l+s+s1}{r}\PY{l+s+s1}{\PYZsq{}}\PY{p}{]}\PY{p}{,} \PY{l+s+s1}{\PYZsq{}}\PY{l+s+s1}{*}\PY{l+s+s1}{\PYZsq{}}\PY{p}{)}
         \PY{n}{decorate}\PY{p}{(}\PY{n}{xlabel} \PY{o}{=} \PY{l+s+s1}{\PYZsq{}}\PY{l+s+s1}{Volume (mL)}\PY{l+s+s1}{\PYZsq{}}\PY{p}{,} \PY{n}{ylabel} \PY{o}{=} \PY{l+s+s1}{\PYZsq{}}\PY{l+s+s1}{Emperical r (sec\PYZca{}\PYZhy{}1)}\PY{l+s+s1}{\PYZsq{}}\PY{p}{,} \PY{n}{title} \PY{o}{=} \PY{l+s+s2}{\PYZdq{}}\PY{l+s+s2}{Collected Data}\PY{l+s+s2}{\PYZdq{}}\PY{p}{)}
\end{Verbatim}


    \begin{center}
    \adjustimage{max size={0.9\linewidth}{0.9\paperheight}}{output_18_0.png}
    \end{center}
    { \hspace*{\fill} \\}
    
    \hypertarget{finding-a-curve-of-best-fit-for-the-data.}{%
\subsubsection{Finding a curve of best fit for the
data.}\label{finding-a-curve-of-best-fit-for-the-data.}}

    \begin{Verbatim}[commandchars=\\\{\}]
{\color{incolor}In [{\color{incolor}17}]:} \PY{c+c1}{\PYZsh{} In order to find a best fit curve, we first linearize the data in data2, then create r\PYZus{}vs\PYZus{}volume\PYZus{}transformed\PYZus{}data }
         \PY{c+c1}{\PYZsh{} as the DataFrame that contains the linearized data.}
         \PY{n}{data2} \PY{o}{=} \PY{p}{\PYZob{}}\PY{l+s+s1}{\PYZsq{}}\PY{l+s+s1}{Volume}\PY{l+s+s1}{\PYZsq{}}\PY{p}{:} \PY{n}{r\PYZus{}vs\PYZus{}volume\PYZus{}data}\PY{p}{[}\PY{l+s+s1}{\PYZsq{}}\PY{l+s+s1}{Volume}\PY{l+s+s1}{\PYZsq{}}\PY{p}{]}\PY{p}{,} \PY{l+s+s1}{\PYZsq{}}\PY{l+s+s1}{Transformed r}\PY{l+s+s1}{\PYZsq{}}\PY{p}{:} \PY{l+m+mi}{1} \PY{o}{/} \PY{n}{r\PYZus{}vs\PYZus{}volume\PYZus{}data}\PY{p}{[}\PY{l+s+s1}{\PYZsq{}}\PY{l+s+s1}{r}\PY{l+s+s1}{\PYZsq{}}\PY{p}{]}\PY{p}{\PYZcb{}}
         \PY{n}{r\PYZus{}vs\PYZus{}volume\PYZus{}transformed\PYZus{}data} \PY{o}{=} \PY{n}{pd}\PY{o}{.}\PY{n}{DataFrame}\PY{p}{(}\PY{n}{data}\PY{o}{=}\PY{n}{data2}\PY{p}{)}\PY{p}{;}
         
         \PY{c+c1}{\PYZsh{}\PYZsh{} Code for plotting the results of the above:}
         
         \PY{c+c1}{\PYZsh{}plot(r\PYZus{}vs\PYZus{}volume\PYZus{}transformed\PYZus{}data[\PYZsq{}Volume\PYZsq{}], r\PYZus{}vs\PYZus{}volume\PYZus{}transformed\PYZus{}data[\PYZsq{}Transformed r\PYZsq{}], \PYZsq{}*\PYZsq{})}
         \PY{c+c1}{\PYZsh{}decorate(xlabel = \PYZsq{}Volume (mL)\PYZsq{}, ylabel = \PYZsq{}Linearized r (sec)\PYZsq{}, title = \PYZdq{}Experimental Data: Linearized r vs. Volume\PYZdq{})}
\end{Verbatim}


    \begin{Verbatim}[commandchars=\\\{\}]
{\color{incolor}In [{\color{incolor}18}]:} \PY{c+c1}{\PYZsh{} Calculate a linear best fit line for the transformed data}
         \PY{n}{x} \PY{o}{=} \PY{n}{r\PYZus{}vs\PYZus{}volume\PYZus{}transformed\PYZus{}data}\PY{p}{[}\PY{l+s+s1}{\PYZsq{}}\PY{l+s+s1}{Volume}\PY{l+s+s1}{\PYZsq{}}\PY{p}{]}
         \PY{n}{y} \PY{o}{=} \PY{n}{r\PYZus{}vs\PYZus{}volume\PYZus{}transformed\PYZus{}data}\PY{p}{[}\PY{l+s+s1}{\PYZsq{}}\PY{l+s+s1}{Transformed r}\PY{l+s+s1}{\PYZsq{}}\PY{p}{]}
         \PY{n}{slope}\PY{p}{,} \PY{n}{intercept}\PY{p}{,} \PY{n}{r\PYZus{}value}\PY{p}{,} \PY{n}{p\PYZus{}value}\PY{p}{,} \PY{n}{std\PYZus{}err} \PY{o}{=} \PY{n}{stats}\PY{o}{.}\PY{n}{linregress}\PY{p}{(}\PY{n}{x}\PY{p}{,}\PY{n}{y}\PY{p}{)}
         
         \PY{c+c1}{\PYZsh{}Create the line from calculated slope and intercept}
         \PY{k}{def} \PY{n+nf}{r\PYZus{}linear\PYZus{}function}\PY{p}{(}\PY{n}{slope}\PY{p}{,} \PY{n}{intercept}\PY{p}{,} \PY{n}{volume}\PY{p}{)}\PY{p}{:}
             \PY{n}{r} \PY{o}{=} \PY{n}{slope} \PY{o}{*} \PY{n}{volume} \PY{o}{+} \PY{n}{intercept}
             \PY{k}{return} \PY{n}{r}
         
         \PY{c+c1}{\PYZsh{}\PYZsh{} Code for plotting the results of the above:}
         
         \PY{c+c1}{\PYZsh{}r\PYZus{}plot = r\PYZus{}linear\PYZus{}function(slope, intercept, r\PYZus{}vs\PYZus{}volume\PYZus{}transformed\PYZus{}data[\PYZsq{}Volume\PYZsq{}])}
         
         \PY{c+c1}{\PYZsh{}def plot\PYZus{}linear\PYZus{}on\PYZus{}data(r\PYZus{}plot):}
         \PY{c+c1}{\PYZsh{}    plot(r\PYZus{}vs\PYZus{}volume\PYZus{}transformed\PYZus{}data[\PYZsq{}Volume\PYZsq{}], r\PYZus{}vs\PYZus{}volume\PYZus{}transformed\PYZus{}data[\PYZsq{}Transformed r\PYZsq{}], \PYZsq{}*\PYZsq{}, label = \PYZdq{}Data\PYZdq{})}
         \PY{c+c1}{\PYZsh{}    plot(r\PYZus{}vs\PYZus{}volume\PYZus{}transformed\PYZus{}data[\PYZsq{}Volume\PYZsq{}], r\PYZus{}plot, label = \PYZdq{}Line of Best Fit\PYZdq{})}
         \PY{c+c1}{\PYZsh{}    decorate(xlabel = \PYZsq{}Volume (mL)\PYZsq{}, ylabel = \PYZsq{}Linearized r\PYZsq{}, title = \PYZdq{}Line of Best fit with Linearized Data\PYZdq{})  }
         \PY{c+c1}{\PYZsh{}plot\PYZus{}linear\PYZus{}on\PYZus{}data(r\PYZus{}plot)}
\end{Verbatim}


    \begin{Verbatim}[commandchars=\\\{\}]
{\color{incolor}In [{\color{incolor}20}]:} \PY{c+c1}{\PYZsh{} De\PYZhy{}linearizes the data and the line of best fit to give the best fit curve, returns the curve:}
         \PY{k}{def} \PY{n+nf}{unflipper}\PY{p}{(}\PY{n}{r\PYZus{}linear\PYZus{}function}\PY{p}{,} \PY{n}{slope}\PY{p}{,} \PY{n}{intercept}\PY{p}{,} \PY{n}{volume}\PY{p}{)}\PY{p}{:}
             \PY{n}{r} \PY{o}{=} \PY{n}{r\PYZus{}linear\PYZus{}function}\PY{p}{(}\PY{n}{slope}\PY{p}{,} \PY{n}{intercept}\PY{p}{,} \PY{n}{volume}\PY{p}{)} \PY{o}{*}\PY{o}{*} \PY{o}{\PYZhy{}}\PY{l+m+mi}{1}
             \PY{k}{return} \PY{n}{r}
         
         \PY{n}{r\PYZus{}nonlinear\PYZus{}function} \PY{o}{=} \PY{n}{unflipper}\PY{p}{(}\PY{n}{r\PYZus{}linear\PYZus{}function}\PY{p}{,} \PY{n}{slope}\PY{p}{,} \PY{n}{intercept}\PY{p}{,} \PY{n}{r\PYZus{}vs\PYZus{}volume\PYZus{}transformed\PYZus{}data}\PY{p}{[}\PY{l+s+s1}{\PYZsq{}}\PY{l+s+s1}{Volume}\PY{l+s+s1}{\PYZsq{}}\PY{p}{]}\PY{p}{)}
         
         \PY{c+c1}{\PYZsh{} Plots the curve of best fit with the data collected:}
         \PY{k}{def} \PY{n+nf}{plot\PYZus{}nonlinear\PYZus{}on\PYZus{}data}\PY{p}{(}\PY{n}{r\PYZus{}nonlinear\PYZus{}function}\PY{p}{)}\PY{p}{:}
             \PY{n}{plot}\PY{p}{(}\PY{n}{r\PYZus{}vs\PYZus{}volume\PYZus{}data}\PY{p}{[}\PY{l+s+s1}{\PYZsq{}}\PY{l+s+s1}{Volume}\PY{l+s+s1}{\PYZsq{}}\PY{p}{]}\PY{p}{,} \PY{n}{r\PYZus{}vs\PYZus{}volume\PYZus{}data}\PY{p}{[}\PY{l+s+s1}{\PYZsq{}}\PY{l+s+s1}{r}\PY{l+s+s1}{\PYZsq{}}\PY{p}{]}\PY{p}{,} \PY{l+s+s1}{\PYZsq{}}\PY{l+s+s1}{*}\PY{l+s+s1}{\PYZsq{}}\PY{p}{,} \PY{n}{label} \PY{o}{=} \PY{l+s+s2}{\PYZdq{}}\PY{l+s+s2}{Data}\PY{l+s+s2}{\PYZdq{}}\PY{p}{)}
             \PY{n}{plot}\PY{p}{(}\PY{n}{r\PYZus{}vs\PYZus{}volume\PYZus{}data}\PY{p}{[}\PY{l+s+s1}{\PYZsq{}}\PY{l+s+s1}{Volume}\PY{l+s+s1}{\PYZsq{}}\PY{p}{]}\PY{p}{,} \PY{n}{r\PYZus{}nonlinear\PYZus{}function}\PY{p}{,} \PY{n}{label} \PY{o}{=} \PY{l+s+s2}{\PYZdq{}}\PY{l+s+s2}{Curve of Best Fit}\PY{l+s+s2}{\PYZdq{}}\PY{p}{)}
             \PY{n}{decorate}\PY{p}{(}\PY{n}{xlabel} \PY{o}{=} \PY{l+s+s1}{\PYZsq{}}\PY{l+s+s1}{Volume (mL)}\PY{l+s+s1}{\PYZsq{}}\PY{p}{,} \PY{n}{ylabel} \PY{o}{=} \PY{l+s+s1}{\PYZsq{}}\PY{l+s+s1}{Emperical r}\PY{l+s+s1}{\PYZsq{}}\PY{p}{,} \PY{n}{title} \PY{o}{=} \PY{l+s+s1}{\PYZsq{}}\PY{l+s+s1}{Curve of Best fit for Collected data}\PY{l+s+s1}{\PYZsq{}}\PY{p}{)}
         
         \PY{n}{plot\PYZus{}nonlinear\PYZus{}on\PYZus{}data}\PY{p}{(}\PY{n}{r\PYZus{}nonlinear\PYZus{}function}\PY{p}{)}
\end{Verbatim}


    \begin{center}
    \adjustimage{max size={0.9\linewidth}{0.9\paperheight}}{output_22_0.png}
    \end{center}
    { \hspace*{\fill} \\}
    
    This curve closely resembles the mathematical one defined earlier, both
in shape and in values - as it should, considering that the mathematical
model was calibrated to match this function.

    \hypertarget{qualitative-result-of-empirical-model}{%
\subsubsection{Qualitative Result of Empirical
Model}\label{qualitative-result-of-empirical-model}}

Now, the simulation is run again with the empirically determined r as a
function of volume replacing the mathematical one:

\emph{Note: The code will be blocked into one cell because it is a
identical repeat of the previously used code, with the exception of
\texttt{unflipper} replacing \texttt{r\_math} and adding in the
additional function arguments as necessary.}

    \begin{Verbatim}[commandchars=\\\{\}]
{\color{incolor}In [{\color{incolor}22}]:} \PY{k}{def} \PY{n+nf}{update\PYZus{}func2}\PY{p}{(}\PY{n}{state}\PY{p}{,} \PY{n}{t}\PY{p}{,} \PY{n}{system}\PY{p}{,} \PY{n}{dvdt}\PY{p}{,} \PY{n}{unflipper}\PY{p}{,} \PY{n}{r\PYZus{}linear\PYZus{}function}\PY{p}{,} \PY{n}{slope}\PY{p}{,} \PY{n}{intercept}\PY{p}{)}\PY{p}{:}
             \PY{n}{unpack}\PY{p}{(}\PY{n}{system}\PY{p}{)}
             
             \PY{n}{T} \PY{o}{=} \PY{n}{state}\PY{o}{.}\PY{n}{T}
             \PY{n}{V} \PY{o}{=} \PY{n}{state}\PY{o}{.}\PY{n}{V}
             
             \PY{c+c1}{\PYZsh{} Empirically determined function of r vs. V}
             \PY{n}{r} \PY{o}{=} \PY{n}{unflipper}\PY{p}{(}\PY{n}{r\PYZus{}linear\PYZus{}function}\PY{p}{,} \PY{n}{slope}\PY{p}{,} \PY{n}{intercept}\PY{p}{,} \PY{n}{state}\PY{o}{.}\PY{n}{V}\PY{p}{)}
                 
             \PY{n}{dTdt} \PY{o}{=} \PY{o}{\PYZhy{}}\PY{n}{r} \PY{o}{*} \PY{p}{(}\PY{n}{T} \PY{o}{\PYZhy{}} \PY{n}{T\PYZus{}env}\PY{p}{)}
             \PY{c+c1}{\PYZsh{} dvdt is passed in as an argument \PYZhy{} it is being swept}
         
             \PY{k}{if}\PY{p}{(}\PY{n}{V}\PY{o}{\PYZgt{}}\PY{l+m+mi}{0} \PY{o+ow}{and} \PY{n}{T} \PY{o}{\PYZgt{}}\PY{l+m+mi}{25}\PY{p}{)}\PY{p}{:}
                 \PY{n}{T} \PY{o}{+}\PY{o}{=} \PY{n}{dTdt} \PY{o}{*} \PY{n}{dt}
                 \PY{n}{V} \PY{o}{+}\PY{o}{=} \PY{n}{dvdt} \PY{o}{*} \PY{n}{dt}
             \PY{k}{else}\PY{p}{:}
                 \PY{n}{V} \PY{o}{=} \PY{l+m+mi}{0}
                 \PY{n}{T} \PY{o}{=} \PY{k+kc}{None}
             \PY{k}{return} \PY{n}{State}\PY{p}{(}\PY{n}{T}\PY{o}{=}\PY{n}{T}\PY{p}{,} \PY{n}{V}\PY{o}{=}\PY{n}{V}\PY{p}{)}
         
         \PY{k}{def} \PY{n+nf}{run\PYZus{}simulation2}\PY{p}{(}\PY{n}{system}\PY{p}{,} \PY{n}{update\PYZus{}func2}\PY{p}{,} \PY{n}{dvdt}\PY{p}{,} \PY{n}{unflipper}\PY{p}{,} \PY{n}{r\PYZus{}linear\PYZus{}function}\PY{p}{,} \PY{n}{slope}\PY{p}{,} \PY{n}{intercept}\PY{p}{)}\PY{p}{:}
             \PY{n}{unpack}\PY{p}{(}\PY{n}{system}\PY{p}{)}
             
             \PY{n}{frame} \PY{o}{=} \PY{n}{TimeFrame}\PY{p}{(}\PY{n}{columns}\PY{o}{=}\PY{n}{init}\PY{o}{.}\PY{n}{index}\PY{p}{)}
             \PY{n}{frame}\PY{o}{.}\PY{n}{row}\PY{p}{[}\PY{l+m+mi}{0}\PY{p}{]} \PY{o}{=} \PY{n}{init}
             \PY{n}{ts} \PY{o}{=} \PY{n}{linrange}\PY{p}{(}\PY{l+m+mi}{0}\PY{p}{,} \PY{n}{t\PYZus{}end}\PY{p}{,} \PY{n}{dt}\PY{p}{)}
             
             \PY{k}{for} \PY{n}{t} \PY{o+ow}{in} \PY{n}{ts}\PY{p}{:}
                 \PY{n}{frame}\PY{o}{.}\PY{n}{row}\PY{p}{[}\PY{n}{t}\PY{o}{+}\PY{n}{dt}\PY{p}{]} \PY{o}{=} \PY{n}{update\PYZus{}func2}\PY{p}{(}\PY{n}{frame}\PY{o}{.}\PY{n}{row}\PY{p}{[}\PY{n}{t}\PY{p}{]}\PY{p}{,} \PY{n}{t}\PY{p}{,} \PY{n}{system}\PY{p}{,} \PY{n}{dvdt}\PY{p}{,} \PY{n}{unflipper}\PY{p}{,} \PY{n}{r\PYZus{}linear\PYZus{}function}\PY{p}{,} \PY{n}{slope}\PY{p}{,} \PY{n}{intercept}\PY{p}{)}
                 
             \PY{c+c1}{\PYZsh{} store the final temperature in T\PYZus{}final}
             \PY{n}{system}\PY{o}{.}\PY{n}{T\PYZus{}final} \PY{o}{=} \PY{n}{get\PYZus{}last\PYZus{}value}\PY{p}{(}\PY{n}{frame}\PY{o}{.}\PY{n}{T}\PY{p}{)}
             
             \PY{k}{return} \PY{n}{frame}
         
         \PY{k}{def} \PY{n+nf}{sweepdVdt\PYZus{}empirical}\PY{p}{(}\PY{n}{system}\PY{p}{)}\PY{p}{:}
             \PY{n}{sweep2} \PY{o}{=} \PY{n}{SweepSeries}\PY{p}{(}\PY{p}{)}
             \PY{n}{dvdt\PYZus{}range} \PY{o}{=} \PY{n}{linspace}\PY{p}{(}\PY{o}{\PYZhy{}}\PY{l+m+mi}{5}\PY{p}{,}\PY{l+m+mi}{0}\PY{p}{,}\PY{n}{system}\PY{o}{.}\PY{n}{n\PYZus{}sweep}\PY{p}{)}
             
             \PY{k}{for} \PY{n}{t} \PY{o+ow}{in} \PY{n+nb}{range}\PY{p}{(}\PY{l+m+mi}{0}\PY{p}{,} \PY{n}{system}\PY{o}{.}\PY{n}{n\PYZus{}sweep}\PY{p}{)}\PY{p}{:}
                 \PY{n}{dvdt} \PY{o}{=} \PY{n}{dvdt\PYZus{}range}\PY{p}{[}\PY{n}{t}\PY{p}{]}
                 \PY{n}{sweep2}\PY{p}{[}\PY{n}{t}\PY{p}{]} \PY{o}{=} \PY{n}{run\PYZus{}simulation2}\PY{p}{(}\PY{n}{system}\PY{p}{,} \PY{n}{update\PYZus{}func2}\PY{p}{,} \PY{n}{dvdt}\PY{p}{,} \PY{n}{unflipper}\PY{p}{,} \PY{n}{r\PYZus{}linear\PYZus{}function}\PY{p}{,} \PY{n}{slope}\PY{p}{,} \PY{n}{intercept}\PY{p}{)}
         
             \PY{k}{return} \PY{n}{sweep2}
         
         \PY{n}{swept2} \PY{o}{=} \PY{n}{sweepdVdt\PYZus{}empirical}\PY{p}{(}\PY{n}{system}\PY{p}{)}\PY{p}{;}
         
         \PY{n}{dvdt\PYZus{}range} \PY{o}{=} \PY{n}{linspace}\PY{p}{(}\PY{o}{\PYZhy{}}\PY{l+m+mi}{5}\PY{p}{,}\PY{l+m+mi}{0}\PY{p}{,}\PY{n}{system}\PY{o}{.}\PY{n}{n\PYZus{}sweep}\PY{p}{)}
         \PY{n}{fig} \PY{o}{=} \PY{n}{plt}\PY{o}{.}\PY{n}{figure}\PY{p}{(}\PY{p}{)}
         \PY{n}{ax} \PY{o}{=} \PY{n}{plt}\PY{o}{.}\PY{n}{subplot}\PY{p}{(}\PY{l+m+mi}{111}\PY{p}{)}\PY{p}{;}
         \PY{k}{for} \PY{n}{i} \PY{o+ow}{in} \PY{n+nb}{range}\PY{p}{(}\PY{n}{system}\PY{o}{.}\PY{n}{n\PYZus{}sweep}\PY{p}{)}\PY{p}{:}
             \PY{n}{dvdt} \PY{o}{=} \PY{n+nb}{round}\PY{p}{(}\PY{n}{dvdt\PYZus{}range}\PY{p}{[}\PY{n}{i}\PY{p}{]}\PY{p}{,} \PY{l+m+mi}{2}\PY{p}{)}
             \PY{n}{ax}\PY{o}{.}\PY{n}{plot}\PY{p}{(}\PY{n}{swept}\PY{p}{[}\PY{n}{i}\PY{p}{]}\PY{o}{.}\PY{n}{index}\PY{p}{,} \PY{n}{swept2}\PY{p}{[}\PY{n}{i}\PY{p}{]}\PY{o}{.}\PY{n}{T}\PY{p}{,} \PY{n}{label} \PY{o}{=} \PY{n+nb}{str}\PY{p}{(}\PY{n}{dvdt}\PY{p}{)} \PY{o}{+} \PY{l+s+s1}{\PYZsq{}}\PY{l+s+s1}{ mL/sec}\PY{l+s+s1}{\PYZsq{}}\PY{p}{)}
             \PY{n}{decorate}\PY{p}{(}\PY{n}{xlabel} \PY{o}{=} \PY{l+s+s1}{\PYZsq{}}\PY{l+s+s1}{Time (s)}\PY{l+s+s1}{\PYZsq{}}\PY{p}{,}
                     \PY{n}{ylabel} \PY{o}{=} \PY{l+s+s1}{\PYZsq{}}\PY{l+s+s1}{Temperature (C)}\PY{l+s+s1}{\PYZsq{}}\PY{p}{,}
                     \PY{n}{title} \PY{o}{=} \PY{l+s+s1}{\PYZsq{}}\PY{l+s+s1}{Empirical Model: Temperature of Coffee as Volume Decreases...}\PY{l+s+s1}{\PYZsq{}}\PY{o}{.}\PY{n}{format}\PY{p}{(}\PY{n}{system}\PY{o}{.}\PY{n}{t\PYZus{}end}\PY{p}{)}\PY{p}{)}
             \PY{n}{ax}\PY{o}{.}\PY{n}{legend}\PY{p}{(}\PY{n}{bbox\PYZus{}to\PYZus{}anchor}\PY{o}{=}\PY{p}{(}\PY{l+m+mf}{1.1}\PY{p}{,}\PY{l+m+mf}{1.05}\PY{p}{)}\PY{p}{)}
\end{Verbatim}


    \begin{center}
    \adjustimage{max size={0.9\linewidth}{0.9\paperheight}}{output_25_0.png}
    \end{center}
    { \hspace*{\fill} \\}
    
    Analysis of qualitative result\ldots{}

    \hypertarget{quantitative-result-of-empirical-model}{%
\subsubsection{Quantitative Result of Empirical
Model}\label{quantitative-result-of-empirical-model}}

    \begin{Verbatim}[commandchars=\\\{\}]
{\color{incolor}In [{\color{incolor}16}]:} \PY{c+c1}{\PYZsh{} Creating the time series in which the last temperatures from each dVdt sweep will be stored.}
         \PY{n}{last\PYZus{}temp2} \PY{o}{=} \PY{n}{TimeSeries}\PY{p}{(}\PY{p}{)}
         
         \PY{k}{def} \PY{n+nf}{determine\PYZus{}last\PYZus{}temp}\PY{p}{(}\PY{n}{last\PYZus{}temp}\PY{p}{,} \PY{n}{system}\PY{p}{,} \PY{n}{swept2}\PY{p}{)}\PY{p}{:}
             \PY{c+c1}{\PYZsh{} Selecting the last values from each sweep by selecting the temperature associated with the point }
             \PY{c+c1}{\PYZsh{} at which the volume first hits 0 mL. If the volume never reaches 0, the last temperature is still selected. }
             \PY{k}{for} \PY{n}{i} \PY{o+ow}{in} \PY{n+nb}{range}\PY{p}{(}\PY{n}{system}\PY{o}{.}\PY{n}{n\PYZus{}sweep}\PY{p}{)}\PY{p}{:}
                 \PY{k}{for} \PY{n}{t} \PY{o+ow}{in} \PY{n+nb}{range}\PY{p}{(}\PY{n}{system}\PY{o}{.}\PY{n}{t\PYZus{}end}\PY{p}{)}\PY{p}{:}
                     \PY{k}{if} \PY{p}{(}\PY{n}{swept2}\PY{p}{[}\PY{n}{i}\PY{p}{]}\PY{o}{.}\PY{n}{V}\PY{p}{[}\PY{n}{t}\PY{p}{]} \PY{o}{==} \PY{l+m+mi}{0} \PY{o+ow}{or} \PY{n}{t} \PY{o}{==} \PY{n}{system}\PY{o}{.}\PY{n}{t\PYZus{}end} \PY{o}{\PYZhy{}} \PY{l+m+mi}{1}\PY{p}{)}\PY{p}{:}
                         \PY{n}{last\PYZus{}temp}\PY{p}{[}\PY{n}{i}\PY{p}{]} \PY{o}{=} \PY{n}{swept2}\PY{p}{[}\PY{n}{i}\PY{p}{]}\PY{o}{.}\PY{n}{T}\PY{p}{[}\PY{n}{t}\PY{o}{\PYZhy{}}\PY{l+m+mi}{1}\PY{p}{]}
                         \PY{k}{break}
             \PY{k}{return} \PY{n}{last\PYZus{}temp}
         
         \PY{n}{last\PYZus{}temp} \PY{o}{=} \PY{n}{determine\PYZus{}last\PYZus{}temp}\PY{p}{(}\PY{n}{last\PYZus{}temp2}\PY{p}{,} \PY{n}{system}\PY{p}{,} \PY{n}{swept2}\PY{p}{)}
         
         \PY{c+c1}{\PYZsh{} To plot the data against dVdt and not against the sweep numbers, the dvdt values and last temperature}
         \PY{c+c1}{\PYZsh{} values will be converted into lists and then plotted against each other.}
         \PY{n}{dvdt\PYZus{}list2} \PY{o}{=} \PY{p}{[}\PY{p}{]}
         \PY{n}{last\PYZus{}temp\PYZus{}list2} \PY{o}{=} \PY{p}{[}\PY{p}{]}
         
         \PY{k}{for} \PY{n}{i} \PY{o+ow}{in} \PY{n+nb}{range}\PY{p}{(}\PY{n+nb}{len}\PY{p}{(}\PY{n}{dvdt\PYZus{}range}\PY{p}{)}\PY{p}{)}\PY{p}{:}
             \PY{n}{dvdt\PYZus{}list2}\PY{o}{.}\PY{n}{append}\PY{p}{(}\PY{n}{dvdt\PYZus{}range}\PY{p}{[}\PY{n}{i}\PY{p}{]}\PY{p}{)}
             \PY{n}{last\PYZus{}temp\PYZus{}list2}\PY{o}{.}\PY{n}{append}\PY{p}{(}\PY{n}{last\PYZus{}temp2}\PY{p}{[}\PY{n}{i}\PY{p}{]}\PY{p}{)}
         
         \PY{k}{def} \PY{n+nf}{plot\PYZus{}final\PYZus{}temp\PYZus{}and\PYZus{}minimum\PYZus{}acceptable\PYZus{}temp}\PY{p}{(}\PY{n}{dvdt\PYZus{}list}\PY{p}{,} \PY{n}{last\PYZus{}temp\PYZus{}list2}\PY{p}{,} \PY{n}{minimum\PYZus{}acceptable\PYZus{}temp\PYZus{}list}\PY{p}{)}\PY{p}{:}
             \PY{n}{plot}\PY{p}{(}\PY{n}{dvdt\PYZus{}list}\PY{p}{,} \PY{n}{last\PYZus{}temp\PYZus{}list2}\PY{p}{,} \PY{n}{label} \PY{o}{=} \PY{l+s+s1}{\PYZsq{}}\PY{l+s+s1}{Final Temperature of Coffee}\PY{l+s+s1}{\PYZsq{}}\PY{p}{)}
             
             \PY{c+c1}{\PYZsh{} The minimum acceptable temperature will be plotted as a line. To achieve this, a list of length }
             \PY{c+c1}{\PYZsh{} len(dvdt\PYZus{}range) will created and contain the minimum acceptable temperature in each index. }
             \PY{n}{minimum\PYZus{}acceptable\PYZus{}temp\PYZus{}list} \PY{o}{=} \PY{n+nb}{list}\PY{p}{(}\PY{p}{)}
             \PY{k}{for} \PY{n}{i} \PY{o+ow}{in} \PY{n+nb}{range}\PY{p}{(}\PY{n+nb}{len}\PY{p}{(}\PY{n}{dvdt\PYZus{}range}\PY{p}{)}\PY{p}{)}\PY{p}{:}
                 \PY{n}{minimum\PYZus{}acceptable\PYZus{}temp\PYZus{}list}\PY{o}{.}\PY{n}{append}\PY{p}{(}\PY{n}{minimum\PYZus{}acceptable\PYZus{}temp}\PY{p}{)}
             \PY{n}{plot}\PY{p}{(}\PY{n}{dvdt\PYZus{}list}\PY{p}{,} \PY{n}{minimum\PYZus{}acceptable\PYZus{}temp\PYZus{}list}\PY{p}{,} \PY{n}{label} \PY{o}{=} \PY{l+s+s1}{\PYZsq{}}\PY{l+s+s1}{Minimum Acceptable Temperature}\PY{l+s+s1}{\PYZsq{}}\PY{p}{)}
             \PY{k}{return} \PY{n}{minimum\PYZus{}acceptable\PYZus{}temp\PYZus{}list}
         
         \PY{n}{plot\PYZus{}final\PYZus{}temp\PYZus{}and\PYZus{}minimum\PYZus{}acceptable\PYZus{}temp} \PY{o}{=} \PY{n}{plot\PYZus{}final\PYZus{}temp\PYZus{}and\PYZus{}minimum\PYZus{}acceptable\PYZus{}temp}\PY{p}{(}\PY{n}{dvdt\PYZus{}list}\PY{p}{,} \PY{n}{last\PYZus{}temp\PYZus{}list2}\PY{p}{,} 
                                                                                                   \PY{n}{minimum\PYZus{}acceptable\PYZus{}temp}\PY{p}{)}
         \PY{n}{decorate}\PY{p}{(}\PY{n}{xlabel} \PY{o}{=} \PY{l+s+s1}{\PYZsq{}}\PY{l+s+s1}{dVdt}\PY{l+s+s1}{\PYZsq{}}\PY{p}{,} 
                  \PY{n}{ylabel} \PY{o}{=} \PY{l+s+s1}{\PYZsq{}}\PY{l+s+s1}{Final Temperature (C)}\PY{l+s+s1}{\PYZsq{}}\PY{p}{,} 
                  \PY{n}{title} \PY{o}{=} \PY{l+s+s1}{\PYZsq{}}\PY{l+s+s1}{Final temperature of coffee after }\PY{l+s+si}{\PYZob{}\PYZcb{}}\PY{l+s+s1}{ seconds}\PY{l+s+s1}{\PYZsq{}}\PY{o}{.}\PY{n}{format}\PY{p}{(}\PY{n}{system}\PY{o}{.}\PY{n}{t\PYZus{}end}\PY{p}{)}\PY{p}{)}
\end{Verbatim}


    \begin{center}
    \adjustimage{max size={0.9\linewidth}{0.9\paperheight}}{output_28_0.png}
    \end{center}
    { \hspace*{\fill} \\}
    
    \hypertarget{interpretation}{%
\section{Interpretation}\label{interpretation}}

    insert interpretation

    \hypertarget{abstract}{%
\section{Abstract}\label{abstract}}

    insert abstract


    % Add a bibliography block to the postdoc
    
    
    
    \end{document}
